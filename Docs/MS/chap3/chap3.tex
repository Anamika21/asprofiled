% ********** Rozdzia� 2 **********
\chapter{Profilowanie}
 
\section{Profilowanie � opis}
Profiling, in this document, means monitoring the performance and memory usage of
a program, which is executing on the Common Language Runtime (CLR). This
document details the interfaces, provided by the Runtime, to access such
information. Typically, a very limited audience will use these APIs � developers of
profiling tools.
Just to give the flavor, a typical use for profiling is to measure how much time
(elapsed, or wall-clock, and/or CPU time) is spent within each routine, or within all
code that is executed from a given root routine. To do this, a profiler asks the
Runtime to inform it whenever execution enters or leaves each routine; the profiler
notes the wall-clock and CPU time for each such event, and accumulates the results
at the end of the program. Note that the term routine is being in this document to
indicate a section of code that has an entry point and an exit point. Different
languages use different names for this same concept -- function, procedure, method,
co-routine, subroutine, etc.
Profiling a CLR program requires more support than profiling conventionally compiled
machine code. This is because the CLR has introduced new concepts such as
application domains, garbage collection, managed exception handling, JIT
compilation of code (converting Microsoft Intermediate Language into native machine
code) etc that the existing conventional profiling mechanisms are unable to identify
and provide useful information. The profiling APIs provide this missing information in
an efficient way that causes minimal impact on the performance of the CLR.
Note that JIT-compiling routines at runtime provide good opportunities, as the APIs
allow a profiler to change the in-memory MSIL code stream for a routine, and then
request that it be JIT-compiled anew. In this way, the profiler can dynamically add
instrumentation code to particular routines that need deeper investigation. Although
this approach is possible in conventional scenarios, it�s much easier to do this for the
CLR.




Expose information that existing profilers will require for a user to determine and
analyze performance of a program run on the CLR. Specifically:
.. Common Language Runtime startup and shutdown events
.. Application domain creation and shutdown events
.. Assembly loading and unloading events
.. Module load/unload events
.. Com VTable creation and destruction events
.. JIT-compiles, and code pitching events
.. Class load/unload events
.. Thread birth/death/synchronization
.. Routine entry/exit events
.. Exceptions
.. Transitions between managed and unmanaged execution
.. Transitions between different Runtime contexts
.. Information about Runtime suspensions
Profiling
Page 9
.. Information about the Runtime memory heap and garbage collection activity
� Callable from any COM-compatible language
� Efficient, in terms of CPU and memory consumption � the act of profiling should
not cause such a big change upon the program being profiled that the results are
misleading
� Useful to both sampling and non-sampling profilers. [A sampling profiler inspects
the profilee at regular clock ticks � maybe 5 milliseconds apart, say. A nonsampling
profiler is informed of events, synchronously with the thread that
causes them]


\section{Typy profilowania}
\subsection{Instrumentacja}
\subsection{Pr�bkowanie}
\section{Profilowanie w �rodowisku .NET}
\section{Profilowanie a dynamiczna weryfikacja program�w}
... jaki� tekst\index{tekst} ...


% ********** Koniec rozdzia�u **********
