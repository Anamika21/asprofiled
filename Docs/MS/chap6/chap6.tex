\chapter{Por�wnanie z innymi bibliotekami}
W tej cz�ci opisano dwa rozwi�zania realizuj�ce podej�cie programowania kontraktowego. Podobnie jak biblioteka AsProfiled s� one przeznaczone dla program�w pracuj�cych w ramach platformy .NET.

\section{CodeContracts}
Jest to kompleksowe rozwi�zanie firmy Microsoft, pocz�tkowe tworzone w ramach kom�rki badawczej Mirosoft Research. Obecnie jest ju� dost�pna na rynku w pe�ni funkcjonalna wersja tego zestawu narz�dzi i wchodz� one w sk�ad �rodowiska programistycznego Visual Studio 2010. //

Zasadnicza r�nica, z punktu widzenia u�ytkownika, w stosunku do biblioteki AsProfiled polega na sposobie okre�lania kontrakt�w.
W tym wypadku s� one definiowane jako wywo�ania statycznych metod klasy \emph{Contract}, przy czym kontrakt podawany jako ich argument, w postaci wyra�enia logicznego. Poprawnym wyra�eniem jest dowolne wyra�enie zgodne z regu�ami ich budowania w ramach ustalonego j�zyka programowania.
Podej�cie to ma zasadnicz� zalet�, mianowicie kontrakty s� silnie typowane.
Poni�szy wycinek kodu ilustruje dotychczasowy opis:

\begin{lstlisting}[label=lst:CodeContracts1, caption=CodeContracts - spos�b u�ycia]

public int Test(int arg1, int arg2)
{
 Contract.Requires(PreCondition);
 int result = arg1 * arg2;
 Contract.Ensures(PostCondition);
 return result;
}
\end{lstlisting}

Przy pomocy metod \emph{Requires} i \emph{Ensure} okre�lane s� odpowiednio warunki pocz�tkowe i warunki ko�cowe. Ich argumentem mo�e by� dowolnie skomplikowane wyra�enie obliczalne do warto�ci logicznej, a ka�da z tych metod mo�e by� wywo�ana dowoln� liczb� razy.


\section{LinFu.Contracts}
